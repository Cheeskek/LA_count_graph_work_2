\section{Задание 3. Аналитическое задание множества.}

\textbf{Условие.}

Определите траекторию точки, которая в своем движении остается вдвое ближе к
точке $A(1,0)$, чем к точке $B(4,0)$.

\begin{enumerate}
    \item Сделайте иллюстрацию к условию задачи: введите удобную для решения систему
координат, необходимые обозначения, подпишите известные величины и
соотношения.
    \item Во введенных обозначениях запишите геометрическое свойство множества, для
которого ищется уравнение.
    \item Сведите геометрическое свойство к уравнению.
    \item Изобразите множество по его уравнению.
\end{enumerate}
\vspace{10mm}
\textbf{Решение.}

It is empty but you can fill it!

\textit{Ответ}: It is empty but you can fill it!
\clearpage
